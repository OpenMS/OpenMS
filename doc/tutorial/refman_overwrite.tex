\documentclass[a4paper]{article}
\usepackage{a4wide}
\usepackage{makeidx}
\usepackage{fancyhdr}
\usepackage{graphicx}
\usepackage{multicol}
\usepackage{float}
\usepackage{textcomp}
\usepackage{alltt}
\usepackage{times}
\ifx\pdfoutput\undefined
\usepackage[ps2pdf,
            pagebackref=true,
            colorlinks=true,
            linkcolor=blue
           ]{hyperref}
\usepackage{pspicture}
\else
\usepackage[pdftex,
            pagebackref=true,
            colorlinks=true,
            linkcolor=blue
           ]{hyperref}
\fi
\usepackage{doxygen}
\makeindex
\setcounter{tocdepth}{1}
\renewcommand{\footrulewidth}{0.4pt}
\begin{document}

\begin{titlepage}
\vspace*{7cm}
\begin{center}
{\Large OpenMS Tutorial\\[1ex]\large Version: TODO }\\
\end{center}
\end{titlepage}

\pagenumbering{arabic}

\section{OpenMS Terms}
	
	\input{tutorial_ms_terms}
	\pagebreak

\section{OpenMS Tutorial}

	This tutorial gives an introduction to the OpenMS core datastructures and algorithms.
	It is intended to allow for a quick start in writing your own applications based on
	the OpenMS framework.
	
	The structure of this tutorial is similar to the modules of the class documentation.
	First, the basic concepts and datastructures of OpenMS are explained. The next chapter is
	about the kernel datastructures. These datastructures represent the actual mass spectronomy
	data: raw data, peaks, spectra and maps. In the following chapters the more sophisticated 
	datastructures and algorithms, e.g. used for peak picking, feature finding and identification
	are presended.
	
	\pagebreak
	\input{tutorial_concept}
	\pagebreak
	\input{tutorial_datastructures}
	\pagebreak
	\input{tutorial_kernel}
	\pagebreak
	\input{tutorial_metadata}
	\pagebreak
	\input{tutorial_format}
	\pagebreak
	\input{tutorial_transformations}
	\pagebreak
	\input{tutorial_filtering}
	\pagebreak
	\input{tutorial_analysis}

\end{document}
