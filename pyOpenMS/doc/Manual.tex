
\documentclass[10pt]{article}
% \usepackage{amsmath}
\usepackage{listings}
\usepackage{url}
\setlength{\parindent}{0pt}
\usepackage[parfill]{parskip}

\usepackage{tocloft}% http://ctan.org/pkg/tocloft
\setlength{\cftsubsecnumwidth}{3.5em}% Set length of number width in ToC for \subsection

\title{pyOpenMS }
\author{OpenMS Development Team}
\date{}
\begin{document}
  \maketitle

  pyOpenMS is a set of Python bindings of the C++ OpenMS library. It allows to
  access a large number of objects (150+) and functions (1400+) of the C++
  code directly from Python. 
  The main functions of the library are explained in the first section of this
  manual. A list of all wrapped functions can be found in the appendix of this
  manual.

  Since all functions in Python directly call C++ and their function signature
  usually corresponds to the one in C++, the OpenMS documentation is for most
  cases the most complete and up-to-date reference also and applies directly
  to pyOpenMS. In this manual, only differences to the existing documentation
  will be highlighted and some general usecases will be explained. The link to
  the documentation of the latest release can be found here:
  \url{http://open-ms.sourceforge.net/documentation/}.

  \tableofcontents

  \pagebreak

  The following section will explain the most important functions of pyOpenMS
  in more detail with full examples of Python code that can be directly
  executed.

  \section{File Input/Output}

  pyOpenMS supports file input and output for various formats. These formats
  include \texttt{DTA2D, DTA, EDTA, FeatureXML, Kroenik, MzData, MzIdentML, IdXML, mzML,
  mzXML, PepXML, ProtXML, TraML, XTandemXML}.

\subsection{Common Pattern}

Most file format objects follow the following idiome

\begin{verbatim}
from pyopenms import *
file = FileObject() 
exp = MSExperiment() 
file.load(filename_in, exp)
# process data
file.store(filename_in, exp)
\end{verbatim}

where FileObject can be any of \texttt{DTA2DFile, DTAFile, mzXML, mzML}.

for \texttt{EDTAFile}, you must load and store a \texttt{ConsensusMap} instead
of an \texttt{MSExperiment}.
for \texttt{FeatureXMLFile}, you must load and store a \texttt{FeatureMap} instead
of an \texttt{MSExperiment}.


\subsection{IdXML}
\begin{verbatim}
from pyopenms import *
id_file = IdXMLFile() 
filename_in = "input.IdXML"
filename_out = "output.IdXML"
protein_ids = []
peptide_ids = []
id_file.load(filename_in, protein_ids, peptide_ids)
# process 
id_file.store(filename_out, protein_ids, peptide_ids)
\end{verbatim}

\subsection{PepXML}
\begin{verbatim}
from pyopenms import *
id_file = PepXMLFile() 
filename_in = "input.pep.xml"
filename_out = "output.pep.xml"
protein_ids = []
peptide_ids = []
id_file.load(filename_in, protein_ids, peptide_ids)
# process 
id_file.store(filename_out, protein_ids, peptide_ids)
\end{verbatim}

\texttt{PepXML} also supports loading with an additional parameter specificing
an MSExperiment which contains the retention time corresponding to the peptide
hits (since these may not be stored in a pep.xml file).

\subsection{ProtXML}
\begin{verbatim}
from pyopenms import *
id_file = ProtXMLFile() 
filename_in = "input.prot.xml"
protein_id = ProteinIdentification()
peptide_id = PeptideIdentification()
id_file.load(filename_in, protein_id, peptide_id)
# process 
# storing not supported
\end{verbatim}

\texttt{ProtXML} currently only supports loading of data.

\subsection{MzIdentML}
\begin{verbatim}
from pyopenms import *
id_file = MzIdentMLFile() 
filename_in = "input.IdXML"
filename_out = "output.IdXML"
identification = Identification()
id_file.load(filename_in, identification)
# process 
id_file.store(filename_out, identification)
\end{verbatim}

Alternatively, MzIdentMLFile also provides a function to load (but not store)
data equivalent to IdXML using two empty vectors that will be filled with
\texttt{ProteinIdentification} and \texttt{PeptideIdentification} objects.

\subsection{TraML}
\begin{verbatim}
from pyopenms import *
tramlfile = TraMLFile() 
filename_in = "input.TraML"
targeted_exp = TargetedExperiment()
idtramlfile.load(filename_in, targeted_exp)
# process 
tramlfilefile.store(filename_out, targeted_exp)
\end{verbatim}

\subsection{MzML}
\begin{verbatim}
from pyopenms import *
file = MzMLFile() 
exp = MSExperiment() 
filename_in = "input.mzML"
filename_out = "output.mzML"
file.load(filename_in, exp)
# process 
file.store(filename_in, exp)
\end{verbatim}

\pagebreak
\section{Parameter Handling}

Paramter handling in OpenMS and pyOpenMS is usually implemented through
inheritance from 
m \texttt{DefaultParamHandler} and allow access to parameters through
the \texttt{Param} object. This means, the classes implement the methods
\texttt{getDefaults}, \texttt{getParameters}, \texttt{setParameters} which
allows access to the default parameters, the current parameters and allows to
set the parameters.
\section{Signal Processing}
Most signal processing algorithms follow a similar pattern in OpenMS. 

\section{Appendix}
In this appendix, a complete list of all wrapped functions is given, ordered
by class.

There is nothing here yet, please run \texttt{create\_manual.py} first.



\end{document}
